\documentclass{article}
\usepackage[utf8]{inputenc}
\topmargin=-0.45in
\evensidemargin=0in
\oddsidemargin=0in
\textwidth=6.5in
\textheight=9.0in
\headsep=0.25in
\usepackage{color}

\title{Laboratorio 2}
\author{William Madrid \\ Gustavo Sosa}
\date{5 de agosto de 2019}


\usepackage{natbib}
\usepackage{graphicx}

\begin{document}

\maketitle

\section*{Ejercicio \#1}
Demostrar utilizando induccion\\
\\
Caso base \\
 \\
\begin{math}
\forall\ n.\ n^3\geq n^2\\
\\
n = 0 \\
(0) (0) (0) \geq (0) (0) \\
0 \geq 0 \\
0 = 0\\  
\\
\end{math}
Caso inductivo \\
\\
\begin{math}
n = n + 1\\
(n + 1) (n + 1) (n + 1) \geq (n + 1) (n + 1)\\
(n + 1) (n^2 + 2(1)(n) + 1^2) \geq (n^2 + 2(1)(n) + 1^2)\\
a = (n^2 + 2(1)(n) + 1^2)\\
(n + 1) a \geq a\\
an + (1)a \geq a\\

\end{math}
 
\section*{Ejercicio \#2}
Demostrar utilizando induccion la desigualdad\\

\begin{math}
∀ n. (1 + x)n \geq nx\\
\end{math}
\\
Caso base\\
\\
\begin{math}
n = 0 y x = 0\\
(1 + x)0 \geq (0) (0) \\
(x)0 \geq (0) (0)\\
1 \geq 0\\
\end{math}
Caso inductivo \\
\\
\begin{math}
n + 1 \\ 
(1  + x )^n+1 \geq (n +1) (x)\\ 
(1  + x )^n * (1  + x ) \geq (n +1) (x) \\
1(1  + x )^n + x (1  + x ) \geq nx + x \\
x(1  + x )^n +
\colorbox{yellow}{
\begin{math}
 (1  + x ) \geq nx 
\end{math}
}
+x
\\
x(1 + x)^n \geq x \\
(1 + x)^n \geq 1 \\
\end{math}
\\
Lo marcado en amarillo es la hipotesis inductiva\\
\\



\end{document}
