\documentclass{article}
\usepackage[utf8]{inputenc}
\topmargin=-0.45in
\evensidemargin=0in
\oddsidemargin=0in
\textwidth=6.5in
\textheight=9.0in
\headsep=0.25in
\usepackage{color}

\title{Laboratorio 3}
\author{Gustavo Sosa \\ William Madrid }
\date{13 de agosto de 2019}

\usepackage{natbib}
\usepackage{graphicx}

\begin{document}

\maketitle

\section*{Ejercicio \#1}
\begin{math}
n = [\sigma(\sigma(\sigma(0)))] = 3\\
m = [\sigma(\sigma(\sigma(\sigma(0))))] = 4\\
\\
n\oplus m =\\
\sigma(\sigma(\sigma(0\oplus m)))\\
\sigma(\sigma(\sigma(m)))\\
\sigma(\sigma(\sigma(\sigma(\sigma(\sigma(\sigma(0)))))))
\end{math}

\section*{Ejercicio \#2}
\[
a\otimes b := \left\{
 \begin{array}{l l}
            0 & \mbox{si } a=o \\
            0 & \mbox{si } b=o \\
            a & \mbox{si } b=1 \\
            b & \mbox{si } a=1 \\
            \sum_{0}^{b}a & \mbox{si } b=\sigma(i) \\
            \sum_{0}^{a}b & \mbox{si } a=\sigma(i) \\
        \end{array}
        \right.
\]\\
\\
\\
Caso base\\
b = 1\\
\begin{math}
a=\sigma(n)\\
a\otimes b\\
\sigma(n)\otimes \sigma(0)\\
\sigma(n)\\
\end{math}\\
\\
Caso inductivo \\
\begin{math}
b=\sigma(\sigma(0))\\
a=\sigma(n)=\sigma(\sigma(\sigma(0)))\\
\\
a\otimes b\\
\sigma(n)\otimes \sigma(\sigma(0))\\
\sigma(n)\oplus \sigma(n)\\
\sigma(\sigma(\sigma(0)))\oplus \sigma(\sigma(\sigma(0)))\\
\sigma(\sigma(\sigma(\sigma(\sigma(\sigma(0))))))\\
\end{math}

\section*{Ejercicio \#3}


\begin{itemize}
        \item{$\sigma(\sigma(\sigma(0)))\otimes 0$}\\
        \\
        \begin{math}
        a=\sigma(\sigma(\sigma(0)))\\
        b=0\\
        \\
        a\otimes b = \\
        \sigma(\sigma(\sigma(0)))\otimes 0 =0\\
        \end{math}
        \item{$\sigma(\sigma(\sigma(0)))\otimes \sigma(0)$}\\
        \\
        \begin{math}
        a=\sigma(\sigma(\sigma(0)))\\
        b=\sigma(0)\\
        \\
        a\otimes b = \\
        \sigma(\sigma(\sigma(0)))\otimes \sigma(0)\\
        \sigma(\sigma(\sigma(0)))\\
        \end{math}
        \item{$\sigma(\sigma(\sigma(0)))\otimes\sigma(\sigma(0))$}\\
        \\
        \begin{math}
        a=\sigma(\sigma(\sigma(0)))\\
        b=\sigma(\sigma(0))\\
        \\
        \sigma(\sigma(\sigma(0)))\otimes \sigma(\sigma(0))\\
        \sigma(\sigma(\sigma(0)))\oplus \sigma(\sigma(\sigma(0)))\\
        \sigma(\sigma(\sigma(\sigma(\sigma(\sigma(0))))))\\
        \end{math}
\end{itemize}


\section*{Ejercicio \#4}

\begin{itemize}
        \item{$a\oplus \sigma(\sigma(0))=\sigma(\sigma(a))$}\\
        \\
        Caso base\\
        \\
        {$a = 0 $}\\
        {$0\oplus \sigma(\sigma(0))=\sigma(\sigma(a))$}\\
        {$\sigma(\sigma(0))=\sigma(\sigma(a))$}\\
        \\
        Caso inductivo\\
        \\
        {$a = \sigma(a) $}\\
        {$\sigma(a)\oplus \sigma(\sigma(0))=\sigma(\sigma(\sigma(a)))$}\\
        {$ \sigma(\sigma(0 \oplus \sigma(a)))=\sigma(\sigma(\sigma(a)))$} Por definicion de suma\\
        {$ \sigma(\sigma(\sigma(a)))=\sigma(\sigma(\sigma(a)))$} Por propiedad de suma\\
        \\
        \\
        \\
        \\
        \\
   
     

        \item{$a \otimes b = b \otimes a$}\\
        \\
    
\[
a\otimes c := \left\{
 \begin{array}{l l}
            0 & \mbox{si } a=o \\
            0 & \mbox{si } c=o \\
            a & \mbox{si } c=1 \\
            c & \mbox{si } a=1 \\
           (a \otimes c)\oplus a & \mbox{si } c=\sigma(c)\\
        \end{array}
        \right.
\]\\
        Caso base\\
        \begin{math}
        a=0\\
        a\otimes b = b\otimes a\\
        0\otimes b = b\otimes 0\\
        0  =  0\\
        \end{math}\\
        \\
        Caso inductivo \\
        \begin{math}
        a=\sigma(a)\\
        a\otimes b = b\otimes a\\
        \end{math}\\
       {$a\otimes \sigma(b)  = (a\otimes b) \otimes a$} Por definicion de la multiplicacion \\
        {$(a\otimes b)\oplus b = (b\otimes a) \oplus b$} Por hipotesis inductiva \\
        {$(a\otimes b)\oplus b = (a\otimes b) \oplus b$} \\
   
        
        
        \item{$a \otimes (b \otimes c)=(a\otimes b)\otimes c$}\\
        \\
        Caso base\\
        \\
        \begin{math}
        c=0\\
        a \otimes (b \otimes 0)=(a\otimes b)\otimes 0\\
        a \otimes  0 = 0\\
          0 = 0\\
        \end{math}
         Caso inducctivo\\
         \\
         {$a \otimes (b \otimes c)=(a\otimes b)\otimes c$} Hipotesis inductiva\\
         {$c=\sigma(c)$}\\
         {$a \otimes (b \otimes \sigma(c))=(a\otimes b)\otimes \sigma(c)$} Por definicion de multiplicacion\\
         {$a \otimes(b \oplus \sigma(c))= (a \otimes b)\otimes c \oplus (a \otimes b)$} Por hipotesis inductiva \\
         {$a \otimes(b \oplus \sigma(c))= a \otimes (b\otimes c) \oplus a \otimes b$} La suma es conmutativa\\
         {$a \otimes(b \oplus \sigma(c))= (a \otimes b) \oplus a \otimes (b\otimes c)$} Por factor comun\\
         {$a \otimes(b \oplus \sigma(c))= a \otimes( b \oplus  (b\otimes c))$} Por definicion de multiplicacion\\
         {$a \otimes(b \oplus \sigma(c))= a \otimes(b \oplus \sigma(c))$}\\
         

        
        \item{$(a\oplus b)\otimes c = (a\otimes c) \oplus (b \otimes c)$}\\
        \\
        Caso base\\
        \begin{math}
        c=0\\
        (a\oplus b)=n\\
        n\otimes 0 = a\otimes 0 \oplus b \otimes 0\\
        0 = 0 \oplus 0\\
        0 = 0\\
        \end{math}\\
        \\
        Caso inductivo\\
        {$(a \oplus b)\otimes c= a\otimes c  \oplus  b \otimes c $}\\
          {$c=\sigma(c)$}\\
        {$(a \oplus b)\otimes \sigma(c)= a\otimes c  \oplus  b \otimes c $} Por definicion de multiplicacion\\
        {$(a \oplus b)\otimes c\oplus (a \oplus b)= (a\otimes c)\oplus a  \oplus ( b \otimes c) \oplus b $} La suma es conmutativa\\
        {$(a \oplus b)\otimes c\oplus (a \oplus b)= (a\otimes c)  \oplus ( b \otimes c) \oplus a\oplus b $} Por factor comun\\
        {$(a \oplus b)\otimes c\oplus (a \oplus b)= (a\oplus b)  \otimes c  \oplus a\oplus b $} La suma es comutativa\\
        {$(a \oplus b)\otimes c\oplus (a \oplus b)= (a\oplus b)  \otimes c  \oplus (a\oplus b) $}\\
        
       
\end{itemize}

\begin{math}
\end{math}




\end{document}
