\documentclass{article}
\usepackage{fancyhdr} % Required for custom headers
\usepackage{lastpage} % Required to determine the last page for the footer
\usepackage{extramarks} % Required for headers and footers
\usepackage[usenames,dvipsnames]{color} % Required for custom colors
\usepackage{graphicx} % Required to insert images
\usepackage{listings} % Required for insertion of code
\usepackage{courier} % Required for the courier font
\usepackage{multirow}
\usepackage{hyperref}
\usepackage{amsmath}
\usepackage{amssymb}

% Margins
\topmargin=-0.45in
\evensidemargin=0in
\oddsidemargin=0in
\textwidth=6.5in
\textheight=9.0in
\headsep=0.25in

\linespread{1.1} % Line spacing

\definecolor{MyDarkGreen}{rgb}{0.0,0.4,0.0} % This is the color used for comments
\lstloadlanguages{c} % Load Perl syntax for listings, for a list of other languages supported see: ftp://ftp.tex.ac.uk/tex-archive/macros/latex/contrib/listings/listings.pdf
\lstset{language=[sharp]c, % Use Perl in this example
        frame=single, % Single frame around code
        basicstyle=\small\ttfamily, % Use small true type font
        keywordstyle=[1]\color{Blue}\bf, % Perl functions bold and blue
        keywordstyle=[2]\color{Purple}, % Perl function arguments purple
        keywordstyle=[3]\color{Blue}\underbar, % Custom functions underlined and blue
        identifierstyle=, % Nothing special about identifiers                                         
        commentstyle=\usefont{T1}{pcr}{m}{sl}\color{MyDarkGreen}\small, % Comments small dark green courier font
        stringstyle=\color{Purple}, % Strings are purple
        showstringspaces=false, % Don't put marks in string spaces
        tabsize=5, % 5 spaces per tab
        %
        % Put standard Perl functions not included in the default language here
        morekeywords={rand},
        %
        % Put Perl function parameters here
        morekeywords=[2]{on, off, interp},
        %
        % Put user defined functions here
        morekeywords=[3]{test},
       	%
        morecomment=[l][\color{Blue}]{...}, % Line continuation (...) like blue comment
        numbers=left, % Line numbers on left
        firstnumber=1, % Line numbers start with line 1
        numberstyle=\tiny\color{Blue}, % Line numbers are blue and small
        stepnumber=5 % Line numbers go in steps of 5
}

\newcommand{\horrule}[1]{\rule{\linewidth}{#1}}

% Creates a new command to include a perl script, the first parameter is the filename of the script (without .pl), the second parameter is the caption
\newcommand{\perlscript}[2]{
\begin{itemize}
\item[]\lstinputlisting[caption=#2,label=#1]{#1.cs}
\end{itemize}
}




\usepackage{natbib}
\usepackage{graphicx}

\begin{document}
\begin{tabular}{l l}
\multirow{5}{*}{\includegraphics[width=2cm]{logo.png}}
 & Universidad del Istmo de Guatemala \\
 & Facultad de Ingenieria \\
 & Ing. en Sistemas \\
 & Informatica 1 \\
 & Prof. Ernesto Rodriguez - \href{mailto:erodriguez@unis.edu.gt}{erodriguez@unis.edu.gt} \\
\end{tabular}


\begin{center}
        \horrule{0.5pt}
        \huge{Resolucion del Examen Parcial \#1} \\
        \large{Gustavo Sosa\\ William Madrid} \\
        \horrule{1pt}
\end{center}

\section*{Ejercicio \#1: Induccion (20\%)}

\begin{enumerate}
        \item{$\forall\ n\geq 1.\ 2 \ast n$ es par}\\
        \\
        Caso base\\
        {$n=1 $}\\
        {$2(1)$} es par\\
       R: {$2 $} es par\\
      
        Caso inductivo\\
        {$2\ast n$} es par, hipotesis indutiva\\
        {$ n= n+1$}\\
        {$2(n+1)$} es par\\
        {$2n + 2$} es par\\
        Por hipotesis inductiva 2n es par y 2 es par\\
        R: 2 es par\\
        
        \item{$\forall\ n\geq 4.\ 2^n < n!$, donde $n!= 1\ast 2\ast 3\ast...\ast(n-1)\ast n  $}\\
        \\
        Caso base\\
        {$n=4$}\\
        {$2^4 < 1\ast 2\ast 3\ast 4$}\\
        R: {$16 < 24$}\\
        
    
        Caso inductivo\\
        {$n = n+1$}\\
        hipotesis inductiva\\
        {$\forall\ n\geq 4.\ 2^n < n!$}\\
        {$2^n+1 < (n+1)!$}\\
        {$2^n\ast 2 < n! (n+1)$}\\
        {$2[2^n < n!](n+1)$} {$=>$} Ej: {$2\ast 2^5 < 5! $}
        {$64 < 720 $}\\
        hipotesis inductiva\\
        {$2\leq (n+1)$}\\
        {$1\leq n$}\\
        {$n\geq 4  => n\geq 1$}\\
        
      
\end{enumerate}
\section*{Ejercicio \#2: Induccion (60\%)}
\begin{enumerate}
        \item{La funcion factorial$(n!)$ en donde $n!= 1\otimes 2\otimes 3\otimes...\ast (n-1)\otimes n  $} \\
         \\
\(
n! := \left\{
 \begin{array}{l l}
            1 & \mbox{si } n=0 \\
            (x!)\otimes (\sigma(x)) & \mbox{si } n= \sigma(x) \\
        \end{array}
        \right.
\) \\
        \item{La funcion resta$(-)$ en donde:
        \begin{itemize}
         \item{$a\ominus b = 0$ si $a\leq b$ }
          \item{$a\ominus b = a -b$ de lo contrario}
          \end{itemize}
        }\\
        
 \(
a\ominus b := \left\{
 \begin{array}{l l}
            0 & \mbox{si } a\leq b \\
            a & \mbox{si } b= 0 \\
            \sigma(a-b) & \mbox{si } a =\sigma(x)\wedge a>b \\
        \end{array}
        \right.
\) \\
        
         \item{La funcion sumatoria $\sum_{i}^{n}$ en donde $\sum_{i}^{n}= i\oplus (i\oplus 1)\oplus ...\oplus(n-1)\oplus n.$ En ontras palabras suma los numeros empezamos por i y terminamos en n.}\\
         
          \(
\sum_{i}^{n} := \left\{
 \begin{array}{l l}
            n & \mbox{si } n = i \\
           \sum_{i}^{x}\oplus \sigma(x) & \mbox{si } n= \sigma(x) \\
        \end{array}
        \right.
\) \\
         \item{La funcion exponente $a^b$ en donde $a^b= a\otimes a\otimes a...$(b veces)} \\
          \\
\(
a^b := \left\{
 \begin{array}{l l}
            1 & \mbox{si } b = 0 \\
            0 & \mbox{si } a = 0 \\ 
            a\otimes a & \mbox{si } b = \sigma(i) \\
        \end{array}
        \right.
\) \\
      
\end{enumerate}





\section*{Ejercicio \#3: Induccion (20\%)}
A continuaci\'on se presenta la definici\'on de la suma y multiplicaci\'on de n\'umeros unarios:\\
\\
\(
a\oplus b := \left\{
 \begin{array}{l l}
            a & \mbox{si } b=1 \\
            b & \mbox{si } a=1 \\
            1\oplus(x\oplus b) & \mbox{si } a=\sigma(x) \\
        \end{array}
        \right.
\)
\(
a\otimes b := \left\{
 \begin{array}{l l}
            0 & \mbox{si } a=o\vee b=0 \\
            a & \mbox{si } b=1 \\
            b & \mbox{si } a=1 \\
            1\oplus(x\otimes b) & \mbox{si } a=\sigma(x) \\
        \end{array}
        \right.

\)\\
\\
\setlength{\parindent}{0cm}

Demostrar utilizando induccion que: \( 2 \otimes a = a \oplus a.\) Puede utilizar una definicion alterna (pero equivalente) de la suma o multiplicacion si lo desea. Recuerde de indicar claramente el caso base, el caso inductivo, la hipotesis inductiva y cada paso de la demostracion.\\ 

Caso base\\
{$a = 0$}\\
{$2\otimes 0 = 0\oplus 0$}\\
{$b\otimes 0 = 0$}\\
{$ 0 = 0$}\\

Caso inductivo\\
\\
{$ a =\sigma(x) $}\\
hipotesis inductiva = {$ 2\otimes x = x\otimes x$}\\
{$ b= \sigma(\sigma(0))$}\\
{$ a = \sigma(x)$}\\
{$ \sigma(\sigma(0))\otimes \sigma(x) = \sigma(x) \oplus \sigma(x)$}\\
{$ a_1 = \sigma(i) = b$}\\
{$ i= \sigma(0)$}\\
{$ b_1 = \sigma(x) = a$}\\
\begin{math}
\sigma(x)\oplus (\sigma(0)\otimes \sigma(x)) = \sigma(x\oplus \sigma(x))\\
a_2= \sigma(j)\\
j = 0\\
b_2 = \sigma(x) =a\\

\sigma(x)\oplus \sigma(x)\oplus (0\otimes \sigma(x) ) = \sigma(x\oplus \sigma(x))\\
\sigma(x)\oplus \sigma(x)\oplus 0 = \sigma(x\oplus \sigma(x))\\
\sigma(x\oplus \sigma(x)) = \sigma(x\oplus \sigma(x))\\
\\
k = x\oplus \sigma(x)\\
\sigma(k) = \sigma(k)\\

\end{math}



\end{document}
