\documentclass{article}
\usepackage[utf8]{inputenc}
\topmargin=-0.45in
\evensidemargin=0in
\oddsidemargin=0in
\textwidth=6.5in
\textheight=9.0in
\headsep=0.25in
\title{Hoja de trabajo  1}

\author{William Madrid \\ Gustavo Sosa}
\date{30 de julio de 2019}


\begin{document}

\maketitle

\section*{Ejercicio \#1}
Elaborado en Github en cada computadora.


\section*{Ejercicio \#2}
Conjunto de nodos del grafo \\
(1,1) (1,2) (1,3) (1,4) (1,5) (1,6) (2,2) (2,3) (2,4) (2,5) (2,6) (3,3) (3,4) (3,5) (3,6) (4,4) (4,5) (4,6) (5,5) (5,6) (6,6)\\
\\
Conjunto de vertices del Grafo\\
\setlength{\parindent}{0cm}
$<1,1>$\hspace{1.7cm}$<1,2><1,3><1,4><1,5>$\hspace{0.5cm}$<2,1> <3,1> <4,1> <5,1>$\\
$<1,2><2,1>$\hspace{0.5cm}$<1,1><1,3><1,4><1,6>$\hspace{0.5cm}$<1,1> <3,1> <4,1> <6,1>$\\
$<1,3><3,1>$\hspace{0.5cm}$<1,1><1,2><1,5><1,6>$\hspace{0.5cm}$<1,1> <2,1> <5,1> <6,1>$\\
$<1,4><4,1>$\hspace{0.5cm}$<1,1><1,2><1,5><1,6>$\hspace{0.5cm}$<1,1> <2,1> <5,1> <6,1>$\\
$<1,5><5,1>$\hspace{0.5cm}$<1,1><1,3><1,4><1,6>$\hspace{0.5cm}$<1,1> <3,1> <4,1> <6,1>$\\
$<1,6><6,1>$\hspace{0.5cm}$<1,2><1,3><1,4><1,5>$\hspace{0.5cm}$<2,1> <3,1> <4,1> <5,1>$\\
\noindent

\setlength{\parindent}{0cm}
$<2,2>$\hspace{1.7cm}$<2,1><2,3><2,4><2,6>$\hspace{0.5cm}$<1,2><3,2><4,2><6,2>$\\
$<2,3><3,2>$\hspace{0.5cm}$<2,1><2,2><2,5><2,6>$\hspace{0.5cm}$<1,2><2,2><5,2><6,2>$\\
$<2,4><4,2>$\hspace{0.5cm}$<2,1><2,2><2,5><2,6>$\hspace{0.5cm}$<1,2><2,2><5,2><6,2>$\\
$<2,5><5,2>$\hspace{0.5cm}$<2,1><2,3><2,4><2,6>$\hspace{0.5cm}$<1,2><3,2><4,2><6,2>$\\
$<2,6><6,2>$\hspace{0.5cm}$<2,2><2,3><2,4><2,5>$\hspace{0.5cm}$<2,2><3,2><4,2><5,2>$\\
\noindent

\setlength{\parindent}{0cm}
$<3,3>$\hspace{1.7cm}$<3,1><3,2><3,5><3,6>$\hspace{0.5cm}$<1,3><2,3><5,3><6,3>$\\
$<3,4><4,3>$\hspace{0.5cm}$<3,1><3,2><3,5><3,6>$\hspace{0.5cm}$<1,3><2,3><5,3><6,3>$\\
$<3,5><5,3>$\hspace{0.5cm}$<3,1><3,3><3,4><3,6>$\hspace{0.5cm}$<1,3><3,3><4,3><6,3>$\\
$<3,6><6,1>$\hspace{0.5cm}$<3,2><3,3><3,4><3,5>$\hspace{0.5cm}$<2,3><3,3><4,3><5,3>$\\
\noindent

\setlength{\parindent}{0cm}
$<4,4>$\hspace{1.7cm}$<4,1><4,2><4,5><4,6>$\hspace{0.5cm}$<1,4><2,4><5,4><6,4>$\\
$<4,5><5,4>$\hspace{0.5cm}$<4,1><4,3><4,4><4,6>$\hspace{0.5cm}$<1,4><3,4><4,4><6,4>$\\
$<4,6><6,4>$\hspace{0.5cm}$<4,2><4,3><4,4><4,5>$\hspace{0.5cm}$<2,4><3,4><4,4><5,4>$\\
\noindent

\setlength{\parindent}{0cm}
$<5,5>$\hspace{1.7cm}$<5,1> <5,3> <5,4> <5,6>$\hspace{0.5cm}$<1,5><3,5><4,5><6,5>$\\
$<5,6><6,5>$\hspace{0.5cm}$<5,2><5,3><5,4><5,5>$\hspace{0.5cm}$<2,5><3,5><4,5><5,5>$\\
\noindent

\setlength{\parindent}{0cm}
$<6,6>$\hspace{1.7cm}$<6,2><6,3><6,4><6,5>$\hspace{0.5cm}$<2,6><3,6><4,6><5,6>$\\
\noindent

\section*{Ejercicio \#3}
¿Qué estructura de datos podría representar un lanzamiento de dados?
La estructura de datos adecuada para representar un lanzamiento de dados es grafo debido a que en este se podrán representar todas las posibles combinaciones de cada dado al establecer una relación entre cada elemento de los dos conjuntos de números plasmados en los dados.\\

¿Qué algoritmo podríamos utilizar para generar dicha estructura?
\begin{enumerate}
\item{Al iniciar ambos dados tienen el número 1 en la parte superior.}
\item{Cualquiera de los dados puede rotar 90 grados hacia arriba, abajo, derecha o izquierda. (este se nombrará como dado 2)}
\item{ Después del primer movimiento el dado 2 se rota hacia cualquier dirección anteriormente mencionada hasta que este cubra las 6 caras diferentes finalizando en una diferente a la de inicio.}
\item{Cuando el dado finalice el movimiento hacia todas las caras se puede rotar el dado 1 hacia cualquiera de los lados mencionados.}
\item{El dado 1 se sigue rotando hasta que cubra las 6 caras con la condición de terminar en una cara diferente a la que inició y la que tenga el dado 2}
\item{El dado 2 empieza a moverse de nuevo hasta cubrir todas las caras y termine en una cara diferente a las que ya habían finalizado ambos dados.}
\item{Se repite del paso 4 al 6 hasta que se cumplan todos los nodos posibles.}
\end{enumerate}
¿Cómo nos aseguramos que ese algoritmo siempre produce un resultado?
Debido a que los dados sólo se pueden mover  una cara a la vez, esto provoca que siempre de un resultado sin importar la dirección a la que rote, repitiendose el paso hasta cubrir todas las opciones y así con los siguientes.\\
Además las condiciones escritas en el algoritmo facilitan el cumplimiento de todas las posibles combinaciones entre los dados.
Las condiciones que obligan a que el algoritmo siempre de un resultado son las de rotar solo un dado hasta que haya cubierto todos los posibles resultados y la de al finalizar de cubrir todos los lados con un dado iniciar con el siguiente, las cuales están descritas en los pasos del algoritmo.

\section*{Ejercicio \#4}
La hoja de trabajo ha sido realizada y publicada en Github.

\end{document}
